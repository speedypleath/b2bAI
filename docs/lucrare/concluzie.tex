\section{Concluzii}
\noindent În această lucrare am implementat o colecție de unelte care poate asista muzicienii în procesul de creație, incluzând unelte pentru generarea, mutația, continuarea și combinarea de secvențe muzicale. Acestea sunt configurabile, păstrând astfel expresia creativă a utilizatorului. 
\subsection{Limitări ale aplicației}
\noindent Deși aplicația reprezintă un plugin audio, aceasta nu interacționează deloc cu DAW-ul din care este deschisă; pentru a putea folosi aplicația în cadrul unui DAW este nevoie ca secvențele generate să fie salvate în format MIDI, după care să fie deschise în DAW. \par
În cadrul evaluării fitness-ului unei secvențe velocitatea notelor nu este deloc utilizată, astfel că o dimensiune muzicală care ar putea fi fost folosită în cadrul generării nu este deloc valorificată. În plus, momentan aplicația nu poate genera secvențe polifonice, fiind limitată astfel mulțimea de tipuri de instrumente pentru care pot fi generate secvențe. \par
\subsection{Posibile dezvoltări ulterioare}
\noindent Plugin-ul ar putea interacționa cu DAW-ul în care este deschis folosind mesaje trimise pe MIDI Chanels. Acestea ar putea fi folosite pentru a reda sau a înregistra secvența într-un track. De asemenea, designul interfeței grafice ar putea fi îmbunătățit. \par
Velocitatea ar putea fi folosită pentru a accentua sau diminua momentele de tensiune create de notele consonante sau pentru a schimba nivelul de sincopare al unei secvențe. Secvențele polifonice ar putea fi generate folosind acorduri și progresii de acorduri. Pentru a identifica si valorifica expresiile muzicale dintr-o secvență, generarea ar putea fi implementată folosind optimizare swarm în loc de algoritmi genetici, utilitățiile necesare fiind prezente în modulul deap.
