\appendix
\section{Header files}

\begin{minted}[fontsize=\footnotesize, linenos=true, bgcolor=bg, frame=lines, label=PluginProcessor.h, breaklines=true, escapeinside=@@, baselinestretch=1]{c++}
@\rtask{B2bAIAudioProcessor}@
{
public:
    B2bAIAudioProcessor();
    ~B2bAIAudioProcessor() override;

    void prepareToPlay (double sampleRate, int samplesPerBlock) override;
    void releaseResources() override;

#ifndef JucePlugin_PreferredChannelConfigurations
    bool isBusesLayoutSupported (const 
            AudioProcessor::BusesLayout& layouts) 
            const override;
#endif
    void processBlock (AudioBuffer<float>&, MidiBuffer&) override;
    void saveMidiFile();
    void loadMidiFile(const File& file);
    void loadDirectory(const File& file);
    void updateListBox(const String& text);
    double getTailLengthSeconds() const override;

private:
    MidiFileListBox *midiFileListBox;
    MidiSequence *midiSequence;
    File midiFilesDir;
    AudioProcessorValueTreeState treeState;
    JUCE_DECLARE_NON_COPYABLE_WITH_LEAK_DETECTOR (B2bAIAudioProcessor)
    File getFile(int index);
    void initialiseBuilder(foleys::MagicGUIBuilder &builder) override;
};
\end{minted}

\begin{minted}[fontsize=\footnotesize, linenos=true, bgcolor=bg, frame=lines, label=PianoRollComponent.h, breaklines=true, escapeinside=@@, baselinestretch=1]{c++}
@\rtask{PianoRollComponent}@
class PianoRollComponent: public Component {
private:
    std::unique_ptr<KeyboardComponent> keyboardComponent;
    std::unique_ptr<GridComponent> gridComponent;
    void timerCallback() override;
public:
    enum ColourIds {
        noteColour = 0x00FF00
    };

    PianoRollComponent(MidiKeyboardState& state, 
            KeyboardComponent::Orientation orientation);
    void paint (juce::Graphics&) override;
    void resized() override;
    void mouseWheelMove(const MouseEvent &event, 
    const MouseWheelDetails &wheel) override;
    void setMidiSequence(MidiSequence *sequence);
};

\end{minted}

\begin{minted}[fontsize=\footnotesize, linenos=true, bgcolor=bg, frame=lines, label=MidiFileListBox.h, breaklines=true, escapeinside=@@, baselinestretch=1]{c++}
@\rtask{MidiFileListBox}@
class MidiFileListBox: public ListBoxModel,
                       public Label::Listener,
                       public ChangeBroadcaster,
                       public ChangeListener {
private:
    File midiFilesDir;
    Array<File> midiFiles;
    foleys::SharedApplicationSettings settings;
    String searchText;

    JUCE_DECLARE_NON_COPYABLE_WITH_LEAK_DETECTOR (MidiFileListBox)
public:
    MidiFileListBox();
    ~MidiFileListBox() override;

    void listBoxItemClicked (int rowNumber, const juce::MouseEvent& event) override;
    void listBoxItemDoubleClicked(int row, const juce::MouseEvent &) override;
    void paintListBoxItem (int rowNumber, juce::Graphics &g, int width, int height, bool rowIsSelected) override;

    void labelTextChanged(juce::Label *labelThatHasChanged) override;
    int getNumRows() override;
    void changeListenerCallback (juce::ChangeBroadcaster*) override;
    std::function<void(File file)> onSelectionChanged;
    std::function<void(File file)> onDoubleClick;
    std::function<void(String text)> update;
};

\end{minted}


\begin{minted}[fontsize=\footnotesize, linenos=true, bgcolor=bg, frame=lines, label=SearchBar.h, breaklines=true, escapeinside=@@, baselinestretch=1]{c++}
@\rtask{SearchBar}@
class SearchBar : public foleys::GuiItem
{
private:
    juce::Label label;
    Label::Listener *listener = nullptr;

    JUCE_DECLARE_NON_COPYABLE_WITH_LEAK_DETECTOR (SearchBar)
public:
    FOLEYS_DECLARE_GUI_FACTORY (SearchBar)

    static const juce::Identifier  pText;
    static const juce::Identifier  pJustification;
    static const juce::Identifier  pFontSize;
    static const juce::Identifier  pDestination;

    SearchBar (foleys::MagicGUIBuilder& builder, const juce::ValueTree& node);
    void update() override;
    std::vector<foleys::SettableProperty> getSettableProperties() const override;
    juce::Component* getWrappedComponent() override;
};
\end{minted}

\begin{minted}[fontsize=\footnotesize, linenos=true, bgcolor=bg, frame=lines, label=NoteRectangle.h, breaklines=true, escapeinside=@@, baselinestretch=1]{c++}
@\rtask{NoteRectangle}@
class NoteRectangle: public Rectangle<int> {
public:
    NoteRectangle(int x=0, int y=0, int width=0, int height=0, int p=0);
    NoteRectangle(int p, int v, double s, double e);
    [[nodiscard]] int getPitch() const;
    void setPitch(int pitch);
    friend std::ostream &operator<<(std::ostream &os, const NoteRectangle &note);
    [[nodiscard]] int getVelocity() const;
    void setVelocity(int velocity);
    bool operator==(const NoteRectangle &rhs) const;
    bool operator!=(const NoteRectangle &rhs) const;
    [[nodiscard]] double getStart() const;
    void setStart(double start);
    [[nodiscard]] double getEnd() const;
    void setEnd(double end);
private:
    Note note;
};
\end{minted}

\begin{minted}[fontsize=\footnotesize, linenos=true, bgcolor=bg, frame=lines, label=GridComponent.h, escapeinside=@@, breaklines=true, baselinestretch=1]{c++}
@\rtask{GridComponent}@
class GridComponent: public Component {
private:
    OwnedArray<Range<int>> noteLineRanges;
    OwnedArray<Range<int>> noteRowRanges;
    MidiSequence *notes = nullptr;
    NoteRectangle pressed, new_position;
    NoteRectangle find_note_rect(Point<int> position);
    int normalise(double w, double wMax);
public:
    GridComponent();

    void updateNoteLineRanges(int firstKeyStartPosition);

    void paint(Graphics &g) override;
    void mouseMove(const MouseEvent &event) override;
    void mouseDown(const MouseEvent &event) override;
    void mouseDrag(const MouseEvent &event) override;
    void mouseDoubleClick(const MouseEvent &event) override;
    void mouseUp(const MouseEvent &event) override;
    void setMidiSequence(MidiSequence *sequence);
};
\end{minted}

\begin{minted}[fontsize=\footnotesize, linenos=true, bgcolor=bg, frame=lines, label=PianoRoll.h, escapeinside=@@, breaklines=true, baselinestretch=1]{c++}
@\rtask{PianoRoll}@
class PianoRoll: public foleys::GuiItem {
private:
    PianoRollComponent pianoRoll;

    JUCE_DECLARE_NON_COPYABLE_WITH_LEAK_DETECTOR(PianoRoll)
public:
    FOLEYS_DECLARE_GUI_FACTORY(PianoRoll)

    static const juce::Identifier  pSource;
    static const juce::Identifier  pNoteColor;

    PianoRoll(foleys::MagicGUIBuilder& builder, const juce::ValueTree& node);

    void update() override;

    [[nodiscard]] std::vector<foleys::SettableProperty> getSettableProperties() const override;

    juce::Component* getWrappedComponent() override;
};
\end{minted}

\newpage

\section{Referințe cod} 
\begin{minted}[fontsize=\footnotesize, linenos=true, bgcolor=bg, frame=lines, label=Arbore XML, escapeinside=@@, breaklines=true, baselinestretch=1]{xml}
@\rtask{XML}@
<View background-color="FF4E505F" height="10%" display="flexbox" margin="0"
      padding="0" flex-align-self="stretch" flex-direction="column"
      radius="0" border="0" caption="Search" caption-size="0">
  <SearchBar font-size="16.0" justification="top-left" label-background="212c31"
             text="" max-height="25" height="" flex-align-self="stretch" margin="0"
             padding="1" border="0" label-outline="FF4E505F" background-color="FF4E505F"
             radius="0" caption="SearchBox" caption-size="0" pos-x="0%" pos-y="0%"
             pos-width="100%" pos-height="8.25083%" destination="filetree"/>
  <ListBox list-box-model="filetree" pos-x="0" background-color="FF4E505F"
           pos-y="0%" pos-width="100%" pos-height="10.7527%" padding="2"
           caption="FileTree" caption-size="0"/>
</View>
\end{minted}

\begin{minted}[fontsize=\footnotesize, linenos=true, bgcolor=bg, frame=lines, label=Paint grid, escapeinside=@@, breaklines=true, baselinestretch=1]{c++}
@\rtask{Update}@
void GridComponent::paint(Graphics& g)
{
    int height = getHeight();
    int width = getWidth();
    // Draw the background
    for (int i = 0; i < 8; i += 2) {
        auto shadow = DropShadow(Colours::black, 1, Point<int>(0, 0));
        auto rect = Rectangle<int>(noteRowRanges[i * 4]->getStart(),
                                   0,
                                   noteRowRanges[(i + 1) * 4]->getStart() - noteRowRanges[i * 4]->getStart(),
                                   height);
        shadow.drawForRectangle(g, rect);
    }
    g.setColour(Colours::white);

    // Draw note lines
    for (int i = 0; i < noteLineRanges.size(); i++) {
        auto rect = Rectangle<int>(0, noteLineRanges[i]->getStart(), width, noteLineRanges[i]->getLength());
        if (rect.getBottom() < 0 || rect.getY() >= height)
            continue;
        g.drawRect(rect, 1);
        if (!MidiMessage::isMidiNoteBlack(i))
            continue;
        g.fillRect(rect);
    }

    g.setColour(Colours::lightgrey);
    // Draw beats lines
    for (auto noteRowRange : noteRowRanges) {
        g.fillRect(static_cast<int>(noteRowRange->getStart() - 1), 0, 1, height);
    }

    // draw notes
    g.setColour(Colours::green);
    if(notes)
        for (const auto& note : *notes) {
            if(pressed == note) {
                g.fillRect(new_position.expanded(1));
                continue;
            }

            g.fillRect(note);
        }
}
\end{minted}

\begin{minted}[fontsize=\footnotesize, linenos=true, bgcolor=bg, frame=lines, label=Pybind11, escapeinside=@@, breaklines=true, baselinestretch=1]{c++}
@\rtask{Boost}@
std::list<mg::Note> mg::generate() {
    try {
        pybind11::module_ commands = pybind11::module_::import("midi_generator.commands");
        py::object result = commands.attr("generate")();
        std::list<mg::Note> notes;
        for(const auto& obj: result)
            notes.emplace_back(
                    obj.attr("pitch").cast<int>(),
                    obj.attr("velocity").cast<int>(),
                    obj.attr("start").cast<double>(),
                    obj.attr("end").cast<double>()
                    );

        return notes;
    } catch (py::error_already_set &error) {
        error.discard_as_unraisable(__func__ );
        return {};
    }
}
\end{minted}

\begin{minted}[fontsize=\footnotesize, linenos=true, bgcolor=bg, frame=lines, label=CMake, escapeinside=@@, breaklines=true, baselinestretch=1]{CMake}
@\rtask{CMake}@
cmake_minimum_required(VERSION 3.23)
project(b2bAI VERSION 1.0)

# Find python
set(Python3_ADDITIONAL_VERSIONS 3.10.6)
find_package (Python3 COMPONENTS Interpreter Development)
# Find boost
find_package(Boost 1.79.0 REQUIRED COMPONENTS log python310 unit_test_framework filesystem)
add_definitions(${Boost_DEFINITIONS})
# Build shared library
option(BUILD_SHARED_LIBS "Build using shared libraries" ON)
add_subdirectory(lib/midi_generator)

# Build python package
if(NOT IS_DIRECTORY ${CMAKE_BINARY_DIR}/midi_generator)
    file(COPY midi_generator DESTINATION ${CMAKE_BINARY_DIR} PATTERN ".github" EXCLUDE)
    file(COPY scripts/build.sh DESTINATION ${CMAKE_BINARY_DIR}/scripts)
    file(COPY scripts/create_note.py DESTINATION ${CMAKE_BINARY_DIR}/scripts)
    find_program(BASH bash)
    exec_program(${BASH} ${CMAKE_BINARY_DIR}/scripts ARGS "build.sh" RETURN_VALUE PACKAGE_NOT_BUILT)
    if(PACKAGE_NOT_BUILT)
        message(FATAL_ERROR "Couldn't build package")
    endif()
endif()

#JUCE
add_subdirectory(lib/JUCE EXCLUDE_FROM_ALL)
juce_add_module(lib/foleys_gui_magic)
set_property(GLOBAL PROPERTY JUCE_COPY_PLUGIN_AFTER_BUILD TRUE)
option(JUCE_ENABLE_MODULE_SOURCE_GROUPS "Enable Module Source Groups" ON)
set_property(GLOBAL PROPERTY USE_FOLDERS YES)
# Company settings
set_directory_properties(PROPERTIES JUCE_COMPANY_COPYRIGHT "GNU GENERAL PUBLIC LICENSE Version 3")
set_directory_properties(PROPERTIES JUCE_COMPANY_NAME "brahman")
set_directory_properties(PROPERTIES JUCE_COMPANY_WEBSITE "https://github.com/speedypleath/b2bAI-VST")
set_directory_properties(PROPERTIES JUCE_COMPANY_EMAIL "gheorgheandrei13@gmail.com")

# Plugin
add_subdirectory(vst)
# Tests
enable_testing()
add_subdirectory(tests)
\end{minted}

\begin{minted}[fontsize=\footnotesize, linenos=true, bgcolor=bg, frame=lines, label=Testare, escapeinside=@@, breaklines=true, baselinestretch=1]{c++}
@\rtask{Testare}@
void test_extract_note() {
    pybind11::scoped_interpreter guard{};
    auto locals = py::dict();
    py::exec(R"(
        from note import Note
        test_note = Note(60, 100, 0, 2)
    )", py::globals(), locals);
    try {
        auto src = locals["test_note"];
        midi_generator::Note note;
        note.pitch =  src.attr("pitch").cast<int>();
        note.velocity = src.attr("velocity").cast<int>();
        note.start = src.attr("start").cast<double>();
        note.end = src.attr("end").cast<double>();
        BOOST_TEST(note.pitch == 60);
        BOOST_TEST(note.velocity == 100);
        BOOST_TEST(note.start == 0.0f);
        BOOST_TEST(note.end == 2.0f);
    } catch (py::error_already_set &error) {
        BOOST_TEST(false);
    }
}

void text_embed_note() {
    pybind11::scoped_interpreter guard{};

    py::module_ module = py::module_::import("note");

    auto constructor = module.attr("Note")(60, 100, 0.0f, 2.0f);
    auto locals = py::dict();

    py::exec(R"(
        from note import Note

        def check_note(note: Note):
            if note.pitch == 60 and note.velocity == 100:
                return True
            return False
    )", py::globals(), locals);
    auto check = locals["check_note"];
    auto src = check(constructor);
    BOOST_TEST(src.cast<bool>());
}

test_suite* init_unit_test_suite( int /*argc*/, char* /*argv*/[] )
{
    framework::master_test_suite().
        add(BOOST_TEST_CASE_NAME(&test_extract_note, "extract note"));
    framework::master_test_suite().
        add(BOOST_TEST_CASE_NAME(&text_embed_note, "embed note"));
    return nullptr;
}
\end{minted}

\begin{minted}[fontsize=\footnotesize, linenos=true, bgcolor=bg, frame=lines, label=Mutație, escapeinside=@@, breaklines=true, baselinestretch=1]{python}
@\rtask{mutație}@
def mutation(config: Configuration, genes):
    for gene in genes:
        change = random.random()
        if change < config.pitch_change_rate:
            change_2 = random.random()
            if change_2 > config.consonance_rate:
                gene.pitch = random.choice(config.scale.notes[30:40])
            else:
                gene.pitch = random.choice(list(set(POSSIBLE_NOTES) - set(config.scale.notes))[30:40])

        change = random.random()
        if change < config.length_change_rate:
            gene.remaining_ticks = 0
    return genes,
\end{minted}

\begin{minted}[fontsize=\footnotesize, linenos=true, bgcolor=bg, frame=lines, label=Generator, escapeinside=@@, breaklines=true, baselinestretch=1]{python}
@\rtask{generator}@
def generator(config: Configuration=Configuration()):
    prev_gene: Gene | None = None

    def create_random_gene() -> Gene:
        nonlocal prev_gene
        if prev_gene is not None and prev_gene.remaining_ticks > 1:
            prev_gene = Gene(prev_gene.pitch, prev_gene.velocity, prev_gene.remaining_ticks - 1)
            return prev_gene

        tick = min(config.rate)
        duration = int(random.choice(config.rate) / tick)
        key = random.choice(list(config.scale.notes)[30:40])
        velocity = random.randint(80, 100)
        prev_gene = Gene(key, velocity, duration)
        return prev_gene

    return create_random_gene
\end{minted}

\begin{minted}[fontsize=\footnotesize, linenos=true, bgcolor=bg, frame=lines, label=Wrapper, escapeinside=@@, breaklines=true, baselinestretch=1]{python}
@\rtask{wrapper}@
def check_remaining_ticks():
    def decorator(func):
        def wrapper(*args, **kwargs):
            offspring = func(*args, **kwargs)
            for genes in offspring:
                for i, gene in enumerate(genes):
                    if gene.remaining_ticks == 0:
                        continue
                    gene.remaining_ticks = len(list(
                        takewhile(lambda x: x.pitch == gene.pitch, genes[i:])))
            return offspring
        return wrapper
    return decorator
\end{minted}

\begin{minted}[fontsize=\footnotesize, linenos=true, bgcolor=bg, frame=lines, label=Off-Beatness, escapeinside=@@, breaklines=true, baselinestretch=1]{python}
@\rtask{offbeatness}@
def off_beatness(notes: list[Note]) -> float:
    b = [x for x in filter(lambda i: len(notes) % i == 0, range(1, len(notes) + 1))]
    weights = [0 if len(list(filter(lambda i: x % i == 0, b))) else 1 for x in notes]
    return sum(weights) / len(notes)
\end{minted}

\begin{minted}[fontsize=\footnotesize, linenos=true, bgcolor=bg, frame=lines, label=WNBD, escapeinside=@@, breaklines=true, baselinestretch=1]{python}
@\rtask{WNBD}@
def weighted_note_to_beat(notes: list[Note]) -> float:
    for note in notes:
        left = floor(note.start)
        right = left + 1
        distance = min(note.start - left, abs(note.start - right))

        if distance == 0:
            continue
        if right < note.end < right + 1:
            total += 2 / distance
        else:
            total += 1 / distance

    return total / len(notes)
\end{minted}

\begin{minted}[fontsize=\footnotesize, linenos=true, bgcolor=bg, frame=lines, label=LZ77, escapeinside=@@, breaklines=true, baselinestretch=1]{python}
@\rtask{lz77}@
def encode_lz77(string: list, window_size=100):
    encoded = string[: window_size + 1]
    i = window_size
    while i < len(string) - window_size:
        input_buffer = string[i: i + window_size + 1]
        window = string[i - window_size: i + window_size + 1]

        substring = max([reduce(lambda x, y: x + y,
                                map(lambda x: x[0],
                                    takewhile(lambda x: x[0] == x[1],
                                              zip_longest(string[i + j: i + window_size], input_buffer)
                                              )
                                    )
                                , '') for j in range(-window_size, 0)]
                        , key=len)

        if substring == '':
            i += 1
            encoded += f'0,0${string[i]}'
        else:
            i += len(substring)
            offset = window.find(substring)
            encoded += f'{str(offset)},{str(len(substring))}${string[i]}'

    return encoded
\end{minted}