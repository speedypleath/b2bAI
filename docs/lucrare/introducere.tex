\section{Introducere}
\noindent Compoziția algoritmică este o tehnică folosită frecvent de către muzicieni. Există diverse metode de a folosi compoziția algoritmică în procesul de creație muzicală: procesul poate să fie neasistat (compoziția este în total creată în mod algoritmic fără intervenția unui muzician), sau poate fi centrat în jurul componentei umane; datele generate pot acționa direct asupra undelor sonor, sau pot fi complet abstractizate de modul în care este generat suntetul, precum este generarea algoritmică de partituri muzicale. \par 
În această lucrare voi analiza compoziția algoritmică ca unealtă asistivă în procesul de creație uman și voi încerca să implementez o interfață care să permită generarea parametrizată de secvențe muzicale scurte în format MIDI. În continuarea acestui capitol voi argumenta motivul pentru care consider importantă studierea compoziției algoritmice ca procedeu stilistic și voi prezentă câteva dintre metodele actuale de implementare. 
\par Voi începe cel de-al doilea capitol prin a explica câteva noțiuni fundamentale din teoria muzicală clasică, urmând ca apoi să prezint metode matematice de modelare a noțiunilor prezentate. În continuare voi descrie câteva particularități ale formatului digital MIDI și ale extensiilor folosite în domeniul muzical digital.
\par Capitolul 3 al lucrării va conține prezentarea structurii implementării pe care o propun, precum și uneltele pe care le-am folosit pentru a configura, testa și construi aplicația.
\par Al patrulea capitol explică implementarea componentei grafice. Aici voi ilustra modul în care am utilizat componenta teoretică în cadrul implementării, precum și modul de utilizare și funcționare a plugin-ului. Cel de-al 5-lea capitol al lucrării prezintă implementarea și utilizarea algoritmilor genetici în cadrul compoziției algoritmice.
\par În final voi analiza limitările implementării prezentate, precum și modul în care aceasta poate fi îmbunătățită. Lucrarea este însoțită de două anexe, prima conține prototipurile claselor din cadrul aplicației, a doua conține fragmente de cod folosite în împlementarea componentelor și tehnologiilor diverse utilizate.
\subsection{Motivație}
\noindent Compoziția algoritmică în cadrul creației muzicale este o tehnică utilizată frecvent, care își are originile într-o oarecare formă în antichitate. Conceptul de a utiliza instrucțiuni și procese formale în cadrul compoziției muzicale își are originile în sistemul muzical utilizat în Grecia Antică \cite{website:history}, \cite{history}. Există diferite sisteme muzicale concepute în aceasta perioadă, precum sistemul de acordare Pitagorean, un algoritm care construiește o scară muzicală între două note muzicale cu rația frecvențelor 1:2. Totuși, sistemele de compoziție algoritmică nu au putut sa crească în complexitate semnificativ până la aparația sistemelor automate de calcul. Astfel, apariția calculatoarelor și a instrumentelor electronice a facilitat apariția unor noi dimensiuni muzicale, iar procesul de creație muzicală s-a schimbat fundamental. Spre exemplu, introducerea sintetizatoarelor de sunet, instrumente capabile de a genera și reda un număr nelimitat de frecvențe sonore a introdus dimensiunea complet nouă în procesul compozițional. În prezent, procesul de creație muzicală transcede regulilor uzuale definite de teoria muzicală și nu mai este în mod necesar orientat în jurul lor. Totuși, acestea încă reprezintă în general fundația peste care este construită o compoziție. Consider astfel că automatizarea acestor procese ar facilita o libertate de explorare mai mare în cadrul procesului compozițional, în special în rândul artiștilor neexperimentați.
\subsection{Implementāri existente}
\noindent Există numeroase modele folosite în implementarea sistemelor de compoziție algoritmică. Multe dintre acestea tratează problema prezenetată ca pe o problemă de optimizare, precum modelele Markov \cite{markov}, modelele bazate pe învățare ranforsată \cite{reinforced}, sau modelele evoluționare \cite{genetic_rc}, \cite{genetic_k}. Alte exemple de modele folosite sunt modelele matematice și modelele translaționale.  În implementarea propusă am ales folosirea unui model evoluționar, deoarece consider că operatorii genetici pot fi folosiți și în implementarea altor funcționalități utile în creația muzicală. Pe lânga implementările teoretice există și produse comerciale implementate: \par
\begin{itemize} 
    \item \textbf{Magenta Studio} - dezvoltat de Magenta, este o colecție de unelte muzicale implementată folosind modelele open-source de machine learning dezvoltate de companie \cite{magentastudio}. Exemple de funcționalități incluse în cadrul colecției sunt \textit{Generate}, \textit{Interpolate}, sau \textit{Drumify}. 
    \item \textbf{Bassline Studio} - \cite{website:reason} dezvoltat de Reason Studios, este un sequencer folosit pentru a genera secvențe monofonice de note \textit{reason}. Acesta este un plugin "proprietary", însă metoda prin care generează secvențe muzicale pare implementată folosind un model matematic deterministic.
\end{itemize}
    
\subsection{Contribuția personală}
    \noindent Contribuția personală adusă în cadrul acestei lucrări constă în implementarea unui plugin audio și al unui modul pe python pentru manipularea și generarea automată de fișiere MIDI. Codul sursă al plugin-ului este disponibil pe \href{https://github.com/speedypleath/b2bAI}{GitHub}. Modulul de python este publicat prin \href{https://pypi.org/project/midi-generator/}{PyPi}, iar codul sursă al acestuia este disponibil într-un alt repository de  \href{https://github.com/speedypleath/midi_generator}{GitHub}.