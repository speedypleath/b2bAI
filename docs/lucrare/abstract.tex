\begin{abstractpage}

\begin{abstract}{romanian}
Domeniul creativității artificiale a propus sisteme de compoziție algoritmică capabile să producă rezultate impresionante. Totuși, majoritatea prezintă modele autonome, care omit interacțiunea umană din implementare. Ca rezultat, există puține unelte bazate pe compoziție algoritmică accesibile consumatorului obișnuit. Obiectivul acestei lucrări este implementerea unui sistem interactiv de unelte bazate pe algoritmi genetici și operatorii folosiți în cadrul acestora. Produsul final este împachetat sub forma unui audio plugin pentru a putea fi folosit într-un DAW și este astfel mai accesibil și ușor de folosit de către muzicieni.
\end{abstract}

\begin{abstract}{english}
The field of computational creativity has proposed algorithmic composition systems capable of producing impressive results. However, most of them are based on autonomous models, which function without ongoing human intervention. Consequently, there are few examples of algorithmic composition tools accessible to musicians. This paper proposes an interactive system driven by genetic algorithms and the genetic operators they derive. The final product is packaged as an audio plugin and can be opened by most DAWs, resulting in better accessibility and ease of use.
\end{abstract}

\end{abstractpage}